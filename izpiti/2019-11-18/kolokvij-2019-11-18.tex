\documentclass[arhiv]{../izpit}
\usepackage{fouriernc}
\usepackage{xcolor}
\usepackage{tikz}
\usepackage{fancyvrb}
\VerbatimFootnotes{}

\begin{document}
	
	\izpit{Programiranje I: kolokvij}{18.\ november 2019}{
		Čas reševanja je 60 minut.
		Veliko uspeha!
	}
	
	%%%%%%%%%%%%%%%%%%%%%%%%%%%%%%%%%%%%%%%%%%%%%%%%%%%%%%%%%%%%%%%%%%%%%%%
	
	\naloga
	
	\podnaloga Definirajte funkcijo \verb|is_root|, ki sprejme dve celi števili in preveri, ali je prvo število kvadratni koren drugega. 
	
	\begin{verbatim}
	# is_root 10 100;;
	- : bool = true
	# is_root (-2) 4;;
	- : bool = false
	\end{verbatim}
	
	\podnaloga Definirajte funkcijo \verb|pack3|, ki sprejme tri argumente in vrne njihovo trojico.
	\begin{verbatim}
	# pack3 1 false [];;
	- : int * bool * 'a list = (1, false, [])
	\end{verbatim}
	
	\podnaloga Definirajte funkcijo \verb|sum_if_not : (int -> bool) -> int list -> int|, ki vrne vsoto vseh elementov, ki ne ustrezajo danemu predikatu. Za vse točke mora biti funkcija repno rekurzivna.
	
	\begin{verbatim}
	# sum_if_not ((=) 3) [1;2;3;4;5;-3];;
	- : int = 9
	\end{verbatim}
	
	\podnaloga Definirajte funkcijo \verb|apply : ('a -> 'b) list -> 'a list -> 'b list list|, ki vrne seznam seznamov, ki ga dobimo tako, da vse funkcije iz prvega seznama uporabimo na vseh elementih drugega seznama. Za vse točke mora biti funkcija repno rekurzivna.
	
	\begin{verbatim}
	# apply [(+) 1; (-) 2; ( * ) 3] [1; 2; 3];;
	- : int list list = [[2; 1; 3]; [3; 0; 6]; [4; -1; 9]]
	# apply [(<) 1; (=) 2; (>) 3] [1; 2; 3; -1];;
	- : bool list list =
	[[false; false; true]; 
	 [true;  true;  true];
	 [true;  false; false]; 
	 [false; false; true]]
	\end{verbatim}

	
	\naloga
	Delavnik zaposlenega na fakulteti lahko predstavimo s seznamom srečanj, od katerih je vsako predstavljeno z imenom predmeta, vrsto (predavanja/vaje) ter številom šolskih ur. Urnik predstavimo kot seznam seznamov srečanj (pri čemer privzamemo, da ima teden na FMF-ju poljubno število dni).
	
	\podnaloga 
	Definirajte:
	\begin{itemize}
		\item variantni tip \verb|vrsta_srecanja| s konstruktorjema \verb|Predavanja| in \verb|Vaje|,
		\item zapisni tip \verb|srecanje| s polji \verb|predmet|, \verb|vrsta| ter \verb|trajanje|,
		\item tip \verb|urnik| kot okrajšavo za \verb|srecanje list list|.
	\end{itemize}

	\podnaloga Definirajte primera srečanj \verb|vaje|, ki predstavlja tri ure vaj pri Analizi 2a, ter \verb|predavanja|, ki predstavlja dve uri predavanj pri Programiranju 1.
	
	\podnaloga Definirajte primer \verb|urnik_profesor : urnik|, ki ima prvi dan dve uri vaj, tretji dan eno uro predavanj, šesti dan ponovno eno uro vaj, preostanek tedna pa ima prosto. Imena predmetov lahko izberete poljubno.
	
	\podnaloga Defnirajte funkcijo \verb|je_preobremenjen : urnik -> bool|, ki za podani urnik preveri, da so v vsakem dnevu kvečjemu štiri ure predavanj in štiri ure vaj.
	
	\begin{verbatim}
	# je_preobremenjen urnik_profesor;;
	- : bool = false
	\end{verbatim}
	
	\podnaloga Za vsako uro vaj profesor dobi 1 evro, za vsako uro predavanj pa 2 evra. Definirajte funkcijo \verb|bogastvo : urnik -> int|, ki sprejme urnik in izračuna plačilo za profesorja. Funkcija naj bo repno rekurzivna.
	
	\begin{verbatim}
	# bogastvo urnik_profesor;;
	- : int = 5
	\end{verbatim}
	
	
\end{document}